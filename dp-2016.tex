% -*- TeX-engine: xetex -*-
\documentclass[a4paper]{scrartcl}
% \documentclass[12pt,a4paper]{article}
\pagenumbering{gobble}

% for 0.5cm margins
% \usepackage[left=0.5cm,top=0.5cm,right=0.5cm,bottom=0.5cm,bindingoffset=0cm]{geometry}

\usepackage{libertine}

\usepackage[svgnames,hyperref]{xcolor}
\usepackage{url}
\usepackage{graphicx}

\usepackage{tabulary}
\usepackage{booktabs}

% \usepackage{paralist}
% \usepackage{enumitem}

% monokai code listings

\usepackage[%
backend=biber,
style=numeric-comp,
maxbibnames=10,
sorting=none,
sortcites=true,
url=true,
doi=true
% refsegment=chapter,
% ibidtracker=strict
]{biblatex}

% small references
\renewcommand{\bibfont}{\normalfont\small}

% \AtEveryBibitem{\clearfield{month}}
% \AtEveryCitekey{\clearfield{month}}

\addbibresource{dp-2016.bib}

\usepackage[english=british,threshold=15,thresholdtype=words]{csquotes}
\SetCiteCommand{\parencite}

\usepackage[%
unicode=true,
hyperindex=true,
bookmarks=true,
colorlinks=true, % change to false for final
pdfborder=0,
allcolors=DarkBlue,
% plainpages=false,
pdfpagelabels,
hyperfootnotes=true]{hyperref}

\author{Ben Swift, Henry Gardner, John Grundy, Andrew Sorensen}
\date{\today}
\title{DP '16 Application}

% hyphenation
\hyphenation{ARC}

\begin{document}

\renewcommand{\thesection}{\Alph{section}}

\setcounter{section}{3} % C1.
\subsection{Project Description}
\label{sec:project-description}
% (Please upload a Project Description as detailed in the Instructions
% to Applicants in no more than 10 A4 pages and in the required
% format. Please refer to the Instructions to Applicants for detailed
% instructions on the required content and format of the Project
% Description.)

\subsubsection*{PROJECT TITLE}

\subsubsection*{AIMS AND BACKGROUND}

The modern world is saturated with computing power, with contemporary
smart phones outperforming the original Cray~1 supercomputer by three
orders of magnitude. Access to the world's fastest supercomputing
systems has never been easier and business users are now routinely
hiring supercomputing time on cloud-based systems.

Modern computing devices are also spectacularly interactive enabling
users to slide, tap, watch, speak, listen with real-time feedback. The
benefits of interactivity in programming (while long espoused by lisp
programmers) have also gained more widespread acceptance, through
Agile\parencite{Fowler2001} and eXtreme\parencite{Beck1999}
programming practices, and the emerging field of live
programming\parencite{Swift2013b}.

But is all of this power, interactivity and networked connectivity
working together to tackle cutting-edge computing challenges of the
21st century? Consider the following scenarios:
\begin{enumerate}
\item A computational physicist submits a job to the (such and
  such) supercomputer which takes 3 weeks to execute. During
  subsequent data analysis, it is found that there was a bug in the
  specifications of 3 parameters out of 200 in the input deck for this
  simulation.

\item A bioinformatics researcher trawls through millions of generated
  phylogenic candidates in order to match and develop evolutionary
  trees. A subsequent analysis of the results finds that the stopping
  criteria were satisfied after only the first hundered such files had
  been compared.

\item The data generated in a HPC simulation is so large that it
  cannot be moved and needs to be analysed in-situ across one million
  cores only a few of which will contain interesting simulation data.
  The tool used to find the important cores misses details at the
  finest scales.
\end{enumerate}

Due to a combination of technical and social factors, scientific
computing is still largely a batch-compile-run-analyse workflow, and
although there have been calls for more
interactivity\parencite{Gil2007} support for this workflow is very
limited at present\parencite{Mattoso}.

The aim of this project is to bring human-in-the-loop interactivity to
high-performance scientific computing (HPSC), enabling scientists and
programmers to be more productive and take better advantage of the
modern computing landscape to do good science.

\subsubsection*{RESEARCH PROJECT}

\parencite{Mattoso}

\begin{itemize}
\item increased \textbf{interactivity} in software/programming
  environments, with all the associated benefits to programmer
  productivity and effectiveness; and
\item 
\end{itemize}

Computational steering: how's that working out for you?


\subsubsection*{ROLE OF PERSONNEL}



\subsubsection*{RESEARCH ENVIRONMENT}



\subsubsection*{COMMUNICATION OF RESULTS}

reproducable code/software artefacts

\subsubsection*{MANAGEMENT OF DATA}

Data provenance, reproducibility

\subsubsection*{REFERENCES}

\vskip -2em % filthy hack to get the spacing right

\printbibliography[title=\ ]

\newpage
\setcounter{section}{5} % E1.
\setcounter{subsection}{0}
\subsection{Justification of funding requested from the ARC for the
  duration of the Project}
\label{sec:funding-justification}
% (In no more than 5 A4 pages and within the required format fully
% justify, in terms of need and cost, each budget item requested from
% the ARC. Use the same headings as in the Description column in the
% budget at Part D of this Proposal. )



\newpage
% E2.
\subsection{Details of non-ARC contributions}
\label{sec:non-arc-contributions}
% (In no more than 2 A4 pages and within the required format, provide
% an explanation of how non-ARC contributions will support the
% proposed Project. Use the same headings as in the Description column
% in the budget at Part D of this Proposal. )


\newpage
\setcounter{section}{6}
\setcounter{subsection}{11} % F12. 
\subsection{Research Opportunity and Performance Evidence (ROPE)\\
  Recent significant research outputs and ARC grants (since 2005)}
\label{sec:recent-significant-outputs}
% (Please attach a PDF with a list of your recent significant research
% outputs and ARC grants most relevant to this Proposal (20 pages
% maximum). Please refer to the Instructions to Applicants for the
% required content and formatting.)



\newpage
% F13. 
\subsection{Research Opportunity and Performance Evidence (ROPE)\\
  Ten career-best research outputs}
\label{sec:ten-best-outputs}
% (Please attach a PDF with a list of your 10 career-best research
% outputs, with a brief paragraph for each output explaining its
% significance (5 pages maximum). Please refer to the Instructions to
% Applicants for the required content and formatting.)



\newpage
\setcounter{section}{7} % G1. 
\setcounter{subsection}{0}
\subsection{Research support from sources other than the ARC}
\label{sec:other-research-support}
% (For all participants on this Proposal, please provide details of
% research funding from sources other than the ARC (in Australia and
% overseas) for the years 2014 to 2018 inclusive. That is, list all
% Projects/Proposals/Fellowships awarded or requests submitted for
% funding involving the Proposal participants. Please refer to the
% Instructions to Applicants for submission requirements.)

\end{document}
